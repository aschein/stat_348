\pdfoutput=1
\documentclass{article}

% alptex + small aesthetic tweaks + some extra math defs
%\RequirePackage[l2tabu, orthodox]{nag}

% FONTS
\usepackage[utf8]{inputenc} % allow utf-8 input
\usepackage[T1]{fontenc}    % use 8-bit T1 fonts

% Replace default Latin Modern typewriter with its proportional counterpart
% http://www.tug.dk/FontCatalogue/lmoderntypewriterprop/
\renewcommand*\ttdefault{lmvtt}


%%% OPTION 1 - Fourier Math + New Century Schoolbook + ParaType Sans

% % Import Fourier Math (this imposes its own New Century Schoolbook type)
% % http://www.ctan.org/tex-archive/fonts/fouriernc/
% \usepackage{fouriernc}
% \usepackage{amsmath}
% % Replace with TeX Gyre Schola version of New Century Schoolbook (must scale!)
% % http://www.tug.dk/FontCatalogue/tgschola/
% \usepackage[scale=0.92]{tgschola}
% \usepackage[scaled=0.88]{PTSans}

%% OPTION 2 - MathDesign Math + Bitstream Charter + ParaType Sans

% Import MathDesign (this brings along Bitstream Charter)
% http://www.ctan.org/tex-archive/fonts/mathdesign/
\usepackage[bitstream-charter]{mathdesign}
\usepackage{amsmath}
\usepackage[scaled=0.92]{PTSans}


% %%% OPTION 3 - MTPRO 2 Math + Termes Times + ParaType Sans

% \usepackage{tgtermes}
% \usepackage{amsmath}
% \usepackage[subscriptcorrection,
%             amssymbols,
%             mtpbb,
%             mtpcal,
%             nofontinfo  % suppresses all warnings
%            ]{mtpro2}
% \usepackage{scalefnt,letltxmacro}
% \LetLtxMacro{\oldtextsc}{\textsc}
% \renewcommand{\textsc}[1]{\oldtextsc{\scalefont{1.10}#1}}
% \usepackage[scaled=0.92]{PTSans}

% GEOMETRY
\usepackage[
  paper  = letterpaper,
  left   = 1.25in,
  right  = 1.25in,
  top    = 1.0in,
  bottom = 1.0in,
  ]{geometry}

% COLOR
\usepackage[usenames,dvipsnames,table]{xcolor}
%\usepackage[usenames,table]{xcolor}
\definecolor{shadecolor}{gray}{0.9}

% SPACING and TEXT
\usepackage[final,expansion=alltext]{microtype}
\usepackage[english]{babel}
\usepackage[parfill]{parskip}
\usepackage{afterpage}
\usepackage{framed}

%redefine the leftbar environment to accept a width and coloring options
\renewenvironment{leftbar}[1][\hsize]
{%
  \def\FrameCommand
  {%
    {\color{Gray}\vrule width 3pt}%
    \hspace{10pt}%
    %\hspace{0pt}\fboxsep=\FrameSep\colorbox{black!10}%
  }%
  \MakeFramed{\hsize#1\advance\hsize-\width\FrameRestore}%
}%
{\endMakeFramed}

% define a paragraph header function
\DeclareRobustCommand{\parhead}[1]{\textbf{#1}~}

% EDITING
% line numbering in left margin
\usepackage{lineno}
\renewcommand\linenumberfont{\normalfont
                             \footnotesize
                             \sffamily
                             \color{SkyBlue}}
% ragged paragraphs in right margin
\usepackage{ragged2e}
\DeclareRobustCommand{\sidenote}[1]{\marginpar{
                                    \RaggedRight
                                    \textcolor{Plum}{\textsf{#1}}}}
% paragraph counter in right margin
\newcommand{\parnum}{\bfseries\P\arabic{parcount}}
\newcounter{parcount}
\newcommand\p{%
    \stepcounter{parcount}%
    \leavevmode\marginpar[\hfill\parnum]{\parnum}%
}
% paragraph helper
\DeclareRobustCommand{\PP}{\textcolor{Plum}{\P} }

% COUNTERS
\renewcommand{\labelenumi}{\color{black!67}{\arabic{enumi}.}}
\renewcommand{\labelenumii}{{\color{black!67}(\alph{enumii})}}
\renewcommand{\labelitemi}{{\color{black!67}\textbullet}}

% FIGURES
\usepackage{graphicx}
\usepackage{wrapfig}
\usepackage[labelfont=bf,font=footnotesize,width=.9\textwidth]{caption}
\usepackage[format=hang]{subcaption}

% TABLES
\usepackage{booktabs,multirow,multicol}       % professional-quality tables

% ALGORITHMS
\usepackage[algoruled]{algorithm2e}
\usepackage{listings}
\usepackage{fancyvrb}
\fvset{fontsize=\normalsize}

% BIBLIOGRAPHY
%\usepackage{natbib}

% HYPERREF
\usepackage[colorlinks,linktoc=all]{hyperref}
\usepackage[all]{hypcap}
\hypersetup{citecolor=MidnightBlue}
\hypersetup{linkcolor=MidnightBlue}
\hypersetup{urlcolor=MidnightBlue}

% CLEVEREF must come after HYPERREF
\usepackage[nameinlink,capitalise]{cleveref}
\Crefname{section}{\S}{\S}

% ACRONYMS
\usepackage[acronym,smallcaps,nowarn]{glossaries}
% \makeglossaries

% COLOR DEFINITIONS
\newcommand{\red}[1]{\textcolor{BrickRed}{#1}}
\newcommand{\orange}[1]{\textcolor{BurntOrange}{#1}}
\newcommand{\green}[1]{\textcolor{OliveGreen}{#1}}
\newcommand{\blue}[1]{\textcolor{MidnightBlue}{#1}}
\newcommand{\gray}[1]{\textcolor{black!60}{#1}}

% LISTINGS DEFINTIONS
\lstdefinestyle{mystyle}{
    commentstyle=\color{OliveGreen},
    keywordstyle=\color{BurntOrange},
    numberstyle=\tiny\color{black!60},
    stringstyle=\color{MidnightBlue},
    basicstyle=\ttfamily,
    breakatwhitespace=false,
    breaklines=true,
    captionpos=b,
    keepspaces=true,
    numbers=left,
    numbersep=5pt,
    showspaces=false,
    showstringspaces=false,
    showtabs=false,
    tabsize=2
}
\lstset{style=mystyle}

% bibtex import + some code to strip away useless bib info (volume number, isbn, and the ilk), and to standardize capitalization
% warning: the arxiv uses an outdated bibtex, which causes cryptic and frustrating upload errors 
% easiest solution: install texlive 2016 from ftp://tug.org/historic/systems/texlive/ (download the ISO) and compile using the texlive 2016 pdflatex and bibtex
% alternatively, look upon https://github.com/plk/biblatex/wiki/biblatex-and-the-arXiv and despair. 

%********************************************************************
% Extra definitions
%********************************************************************
\usepackage{natbib,wrapfig}
\usepackage{enumitem} % tight enumerates
\usepackage[separate-uncertainty=true,multi-part-units=single]{siunitx} % better table control
\usepackage{xcolor}
\usepackage{booktabs}
\usepackage[affil-it]{authblk}

\usepackage{titling}
\setlength{\droptitle}{-2.5cm}

\title{Assignment 3: Exponential families, conjugacy, and entropy}
\date{} 
\author{Stat 348, Spring 2023}


\begin{document}
\maketitle

% \begin{abstract}
% \end{abstract}
\vspace{-2em}
\textbf{Instructions:} Submit a well-formatted LaTeX document with your answers, using the same (sub)section names as this document. Show your steps---your answers should contain complete proofs.

\textbf{Due:} Monday, April 17 at 11:59PM on GradeScope.

\section*{Problem 1: Exponential-gamma conjugacy}
For this problem, use the following facts.

The exponential distribution with rate $\mu>0$ has PDF
\begin{align}
\label{eq:expon}
P(x \mid \mu) = \textrm{Expon}(x; \mu) =  \mu \exp(-\mu x), \,\,\, x>0
\end{align}

The gamma distribution with shape $a>0$ and rate $b>0$ has PDF
\begin{align}
\label{eq:gamma}
P(x \mid a, b) = \textrm{Gam}(x; a,\,b) = \frac{b^a}{\Gamma(a)} x^{a-1} \exp(-bx), \,\,\, x>0
\end{align}

In general, an exponential family distribution takes the following form:
\begin{align}
\label{eq:expfam}
P(x \mid \eta) = h(x) \exp\left(\eta^\top t(x) - a(\eta)\right)
\end{align}
where $h(x)$ is the base measure, $\eta$ are the natural parameters, $t(x)$ are the sufficient statistics, and $a(\eta)$ is the log-normalizer.

A conjugate prior for $\eta$ is also exponential family and takes the following form
\begin{align}
\label{eq:conjugate}
P(\eta \mid \lambda) = h_c(\eta) \, \exp\left(\lambda_1^\top \eta - \lambda_0 \, a_\ell(\eta) -a_c(\lambda)\right)
\end{align}
where $\lambda = [\lambda_1, \lambda_0]$ are the natural parameters for the conjugate prior, $t(\eta)=[\eta, -a_\ell(\eta)]$ are its sufficient statistics, and $a_c(\lambda)$ is its log-normalizer. Note that $a_\ell(\eta)$ is the log-normalizer of the likelihood $P(x \mid \eta)$, while $a_c(\lambda)$ is the log-normalizer of the prior $P(\eta \mid \lambda)$.

\subsection*{1.1: Exponential family forms [10 points]} 

Provide exponential family forms for the exponential and gamma distributions. (This means defining $h(x)$, $t(x)$, $\eta$, and $a(\eta)$ and confirming that~\cref{eq:expfam} equals the PDFs above.)

\subsection*{1.2: Conjugacy [10 points]}

Use~\cref{eq:conjugate}, to show that if the likelihood $P(x \mid \mu)$ is exponential (\cref{eq:expon}), then the conjugate prior for $\mu$ is a gamma distribution $P(\mu \mid a, b)$ (\cref{eq:gamma}). 

\subsection*{1.3: Posterior [10 points]}
Provide the form of the gamma posterior $P(\mu \mid x_{1:n}, a, b)$ where $x_{1:n} \equiv x_1, \dots, x_n$, and $x_i \stackrel{\textrm{iid}}{\sim} P(x \mid \mu)$.

\subsection*{1.4: Prior predictive distribution [10 points]}

Derive the prior predictive distribution $P(x_1 \mid a, b) \!=\! \int P(x_1 \mid \mu) \, P(\mu \mid a, b) \,d\mu$ for one data point $x_1$.\looseness=-1

\subsection*{1.5: Posterior predictive distribution [10 points]} 

Derive the posterior predictive distribution $P(x_{n+1} \mid x_{1:n}, a, b) \!=\! \int P(x_{n+1} \mid \mu) \, P(\mu \mid x_{1:n}, a, b) \,d\mu$ for a single new point $x_{n+1}$ conditional on a data set $x_{1:n}$.\looseness=-1

\section*{Problem 2: Gamma-Poisson and entropy}
For this problem, use the following facts.

The Poisson distribution with rate $\mu>0$ has PMF
\begin{equation}
P(x \mid \mu) = \textrm{Pois}(x; \mu) = \frac{\mu^x}{x!} \exp(-\mu), \,\,\, x \in \{1,2,\dots\}
\end{equation}

The negative binomial distribution with shape $r>0$ and probability parameter $p \in (0,1)$ has PMF
\begin{equation}
P(x \mid r,\, p) = \textrm{NB}(x; r,\, p) = \frac{\Gamma(x+r)}{x! \Gamma(r)} (1-p)^r p^x, \,\,\, x \in \{1,2,\dots\}
\end{equation}

A gamma-Poisson mixture is equal to a negative binomial
\begin{equation}
\int_{0}^\infty \textrm{Pois}(x; \mu) \,\textrm{Gam}(\mu; a,\,b) \,d\mu = \textrm{NB}(x; a,\, \tfrac{1}{1+b})
\end{equation}

\subsection*{2.1: Posterior [5 points]}
If $x \sim \textrm{Pois}(\mu)$ and $\mu \sim \textrm{Gam}(a, b)$, what is the posterior $P(\mu \mid x, a, b)$?

\subsection*{2.2: Exponential family forms [10 points]} 
Provide an exponential family form for the Poisson distribution.

In addition, provide an exponential family form for the negative binomial distribution \textit{with known $r$}---this means only $p$ is treated as a parameter, while $r$ is treated as a known constant (e.g., like $\pi$ or $e$).

\subsection*{2.3: KL divergence between two Poissons [10 points]} Use the exponential family form of the Poisson to derive the Kullback-Leibler (KL) divergence between two Poisson distributions, $\textrm{KL}(\textrm{Pois}(x; \mu_1) \,||\, \textrm{Pois}(x; \mu_2))$. Use the natural-log ($\ln$) form of KL divergence: 
$\textrm{KL}(P(x) \,||\, Q(x)) = \sum_{x \in \mathcal{X}} P(x) \ln\left[\frac{P(x)}{Q(x)}\right]$.

\subsection*{2.4: Poisson and negative binomial entropies [25 points]}

Define the following three quantities.

The entropy of a negative binomial distribution with shape $a$ and probability parameter $\tfrac{1}{1+b}$:
\begin{align}
H\Big(\textrm{NB}(a, \tfrac{1}{1+b})\Big) &= -\sum_{x=0}^\infty \textrm{NB}(x;\, a, \tfrac{1}{1+b})\, \ln \left[\textrm{NB}(x;\,a, \tfrac{1}{1+b})\right]
% 
\intertext{The entropy of a Poisson distribution with rate $y$:}
% 
H\Big(\textrm{Pois}(y)\Big) &= -\sum_{x=0}^\infty \textrm{Pois}(x; y) \ln \left[\textrm{Pois}(x; y)\right]
% 
\intertext{The conditional entropy of a Poisson, conditioned on a gamma prior over $y$ with shape $a$ and rate $b$:}
% 
H\Big(\textrm{Pois}(y) \mid \textrm{Gam}(a, b)\Big) &= \int_0^\infty H\Big(\textrm{Pois}(y)\Big) \,\, \textrm{Gam}(y; a, b) \,dy 
\end{align}

Show that the entropy of the negative binomial is lower bounded:
\begin{align} 
H\Big(\textrm{NB}(a, \tfrac{1}{1+b})\Big) \geq H\Big(\textrm{Pois}(y) \mid \textrm{Gam}(a, b)\Big).
\end{align}

\end{document}


